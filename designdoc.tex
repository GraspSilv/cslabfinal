\documentclass{article}
\begin{document}
Design Document\\
Frank Cipollone and Jack Magiera \\
November 22, 2013 \\
\begin{description}
\item[Call Graph]---------------------------------------------------------------------------------------\\
\begin{tabular}{|l|r|}
\hline
Function & Calls \\
\hline
main & initializeLevel \\
& performAction \\
& calcNextScreen \\
& drawScreen \\
\hline
calcNextScreen & actorLocs\\
& updateChanges \\
\hline
drawScreen & drawLength \\
& drawArb \\
\hline
drawArb & endCScreen \\
\hline
actorLocs & performAction \\
& actorLocs \\
\hline
performAction & createObject \\
\hline
removeObject & \\
\hline
createObject & \\
\hline
\end{tabular}
\item[Prototypes]--------------------------------------------------------------------------------------
\begin{verbatim}
void initializeLevel(float curScreen[SCR]);
void performAction(char a, int p);
void calcNextScreen(float curScreen[SCR]);
void drawScreen(float curScreen[SCR]);
void drawLengths(float stickman[SIZE]);
void drawArb(float object[410]);
void actorLocs(float curScreen[SCR]);
void createObject(float curScreen[SCR], int num);
void removeObject(float curScreen[SCR], int num);
\end{verbatim}
\item[Psuedocode]--------------------------------------------------------------
\begin{verbatim}
int main(void){
    set floats xmax and ymax to boundaries of screen;
    gfx_open();
    initialize curScreen[SCR], with SCR being max number of items on screen;
    initializeLevel(curScreen);
    while(1):
       initialize char a;
       if (character entered):
           performAction(a);
       calcNextScreen(curScreen);
       drawScreen(curScreen);
       pause for animation;
\end{verbatim}
\item[General Description]-----------------------------------------------------\\
This program will begin by initializing the level. It will read an array of 
numbers from a file.  This array of numbers will describe all the objects
on the screen, with each object separated by an indicator.  All the other
functions change this array.  drawScreen reads this array and draws the
screen that it describes, then there is a pause for the animation.  Different
functions will change the velocities, accelerations, or positions of the
various objects.  There will also be some preset motions that can be called
with performAction, so that when the user types 'd' the user will walk right,
and when the user type 'a' the user will walk left.  calcNextScreen will
put all of this together to change the screen for the next stage of animation.
\end{description}
\end{document}
